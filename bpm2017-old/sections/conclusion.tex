\section{Summary}
\label{sec:bpm2017conclusion}

In this paper, we addressed the problem of uncovering hidden work dependencies from \gls{vcs} logs. The main goal was to provide project managers with knowledge about the artifacts co-evolution in the project. Three perspectives of analysis were considered, evolution of the artifacts over time, dependencies among them and structural organization of the project.Our approach works under the assumptions that repositories reflect the hierarchical structure of the project, project participants commit their work regularly during active working
times and they provide informative comments for the changes done. The approach was implemented as a prototype. A scenario of use was provided showing how the approach can be applied and providing some discussions. We also evaluated our approach in real-world data from open source projects showing the potential of the approach.

In future work, we will improve our evaluation varying for instance the time window, the dependency threshold and consider a study case with project managers. We plan to investigate other types of dependencies between artifacts. Specifically, we are interested in a semantic analysis of the work performed in both artifacts, considering for instance some similarity measures. We also aim to improve the visualization to consider other knowledge extracted, for instance the type of change performed in the aggregate events could be shown associated to the activities in the artifact process.