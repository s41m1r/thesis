%% For language-specific hyphenations etc.
\usepackage[english]{babel}
\usepackage[utf8]{inputenc} % Required for inputting international characters
\usepackage[T1]{fontenc} % Output font encoding for international characters
\usepackage{lmodern}
% For subfigures
\usepackage{subcaption}
%\captionsetup{compatibility=false}
%Import the natbib package and sets a bibliography  and citation styles
\usepackage{natbib}
%\setcitestyle{open={[},authoryear,close={]}}
% For nice links
%\usepackage{url}
%Arbitrary size font selection 
\usepackage{type1cm}         
\usepackage{makeidx}         % allows index generation
%\usepackage{cropmark}		% fixes \printindex
\usepackage{graphicx}        % standard LaTeX graphics tool
% when including figure files
\graphicspath{{figures/}}
\usepackage{multicol}        % used for the two-column index
\usepackage[bottom]{footmisc}% places footnotes at page bottom

\usepackage{newtxtext}       % 
\usepackage{newtxmath}       % selects Times Roman as basic font
\usepackage{fancyhdr}% http://ctan.org/pkg/fancyhdr

\usepackage{hyperref}
\hypersetup{
	pdftitle={Mining software development projects},
	pdfauthor={Saimir Bala},
	pdfkeywords={software development, business process management, visualisation, project mining},
	pdfsubject={mining software development projects},
	bookmarksopen=true,
	bookmarksopenlevel=0,
	hypertexnames=true,
	colorlinks=true,       % false: boxed links; true: colored links
	linkcolor=Blue4,          % color of internal links (change box color with linkbordercolor)
%	%linkbordercolor=RedOrange,    
	citecolor=Blue4,        % color of links to bibliography
%	filecolor=magenta,      % color of file links
	urlcolor=blue,           % color of external links
	pdfstartview={FitV},
	unicode=true,
	breaklinks=true,
}

\urlstyle{same}

% For playing with colors in tabular environments
\usepackage{colortbl}

% Compressed, sorted lists of numerical or partly-numerical citations, as regular text or as superscripts
%\usepackage{cite} 

%% For math symbols, such as \nexists
%\usepackage{amssymb}
%% For more math symbols, such as \mapsfrom
%\usepackage{stmaryrd}
\usepackage{textcomp}
%% For advanced graphics
%%% For including figures, graphicx.sty has been loaded in
%%% elsarticle.cls. If you prefer to use the old commands
%%% please give \usepackage{epsfig}
%% For equations, arrays of equations, defining operator names, etc.
%\usepackage{amsmath}
%% For cursive math
%\usepackage{mathrsfs}
%% For math symbols, such as \nexists
%\usepackage{amssymb}
%%% For math environments, such as "definition"
%%\usepackage{amsthm}
%%\theoremstyle{definition}
%%\newdefinition{definition}{Definition}[section]
%%\newtheorem{theorem}{Theorem}[section]
%% For enumerating the line numbers
%\usepackage[left]{lineno}
% For side notes, missing figures and inline to-do's
\usepackage[textsize=scriptsize,backgroundcolor=orange!50]{todonotes}
%% For specifying kewords and acronyms
\usepackage[nonumberlist,nomain,nopostdot,acronym,toc,nogroupskip,symbols]{glossaries}
%\glsdisablehyper
%% For commenting out some parts of the text
%\usepackage{comment}
%%\usepackage{etoolbox}
%%\makeatletter
%%\patchcmd\@combinedblfloats{\box\@outputbox}{\unvbox\@outputbox}{}{%
%%	\errmessage{\noexpand\@combinedblfloats could not be patched}%
%%}%
%%\makeatother
%
%% For smart references
\usepackage{cleveref}
\crefname{algocf}{algorithm}{Algorithms}
\crefname{figure}{figure}{figures}
\crefname{table}{table}{tables}
\crefname{missingfigure}{fig.}{figs.}
\crefname{lstlisting}{listing}{listings}
%\Crefname{algocf}{Algorithm}{Algorithms}
%% To have "Figure 3(a)" in place of "Figure 3a" and  "Table 3(a)" in place of "Table 3a"
%\captionsetup[subfigure]{subrefformat=simple,labelformat=simple}
%    \renewcommand\thesubfigure{(\alph{subfigure})}
\captionsetup[subtable]{subrefformat=simple,labelformat=simple}
    \renewcommand\thesubtable{(\alph{subtable})}

% TikZ/Pgf advanced graphics
\usepackage{tkz-base}
\usetikzlibrary{decorations.pathmorphing,trees,arrows,shapes,automata}
%Make a tree diagram
\usepackage{forest}
%% To use inline and other fancy list-like environments (e.g., inparaenum)
%\usepackage{paralist}
%% To divide a text line into multiple columns
%\usepackage{multicol}
%% To create good-looking book-style tables
\usepackage{booktabs}
\usepackage[normalem]{ulem}
\useunder{\uline}{\ul}{}
\usepackage{longtable}
%% To play around with list environments
%\usepackage{enumitem}
% To create multirow cells in tables
\usepackage{multirow}
%% To create rotated cells in tables
%\usepackage{rotating}
%%\usepackage{array}
%% To create enumerated lists, whose numbering is reversed
%\usepackage{etaremune}
%%% To make algorithmic nice-looking pseudocode
\usepackage[linesnumbered,ruled,noline]{algorithm2e}
%%% For creating side-notes
%% \usepackage{marginnote}
%% For superimposing symbols over one another within math env.
%\usepackage{mathtools}
%% For strange math symbols like \Dashv
%\usepackage{mathabx}
%% For LaTeX if/then statements
%\usepackage{ifthen}
%% For strike-through cancellations
%\usepackage[normalem]{ulem}
%%% The lineno packages adds line numbers. Start line numbering with
%%% \begin{linenumbers}, end it with \end{linenumbers}. Or switch it on
%%% for the whole article with \linenumbers.
%\usepackage{lineno}
%% For highlighted text
%\usepackage{soul}
\usepackage{float}
%% To put table environments and co. side by side
%%\usepackage{floatrow}
%%\floatsetup[table]{style=plaintop}
%% To add dummy text
%\usepackage{lipsum}
%% To have newlines in cells, with commands such as \makecell or \thead
%\usepackage{makecell}
%% To enable text protrusion
\usepackage{microtype}
%%\usepackage[showframe]{geometry}
\usepackage{xspace}
\usepackage{pifont}
\newcommand{\cmark}{\ding{51}}%
\newcommand{\xmark}{\ding{55}}%
%% For code
\usepackage{listings}
\usepackage[section]{placeins}