% Please add the following required packages to your document preamble:
% \usepackage{booktabs}
% \usepackage{graphicx}
% \usepackage[normalem]{ulem}
% \useunder{\uline}{\ul}{}
\begin{table}[]
\centering
\caption{Identified key difficulties and SDM improvement recommendations }
\label{tab:key-difficulties}
\resizebox{\textwidth}{!}{%
\begin{tabular}{@{}m{1cm}m{3cm}m{1cm}m{6cm}@{}}
\toprule
 &
  Identified key difficulties &
   &
  SDM improvement recommendations \\ \midrule
D1 &
  Adjusting prioritizations of user stories to ensure customer satisfaction when customer priorities changed &
  R1 &
  Currently manager is acting as an intermediary 			between developers and the customer. Management needs to start to 			act as connector that encourages and helps the team to establish 			direct collaborations between them and the customer’s product 			owners (lead users). This would increase the team’s autonomy and 			ability to plan its work and at the same time enable the team to 			better consider the architectural and other technical aspects when 			prioritizing user stories. \\ \\ \hdashline \\
D2 &
  Delayed synchronization of list of specifications in MS Excel &
  R2 &
  The list of specifications that is currently 			managed in MS Excel should be moved to JIRA. This will integrate 			backlog with task planning and task completion administration. \\ \\ \hdashline \\
D3 &
  Unsuccessful management pressure to raise performance &
  R3 &
  Instead of pressure to stop undesirable behaviour we suggest management focuses on rewarding desirable 			behaviour. Thus, we suggest to revamp the reward scheme and make a 			desired but still achievable percentage of tasks/bugs completed in 			time an important metric of remuneration. \\ \\ \hdashline \\
D4 &
  Often exceeded estimated sprint tasks time limits in implementing new code and bug fixing &
  R4 &
  To address this difficulty, we propose using individual developer performance data mined from JIRA logs to 			increase the efficiency of task distribution between the 			developers. The developers who are significantly more efficient at 			bug fixing should predominantly work on bug fixing, while the 			developers who are significantly more efficient at implementing 			new code should predominantly work on new code. Additionally, the 			first, the second and the third improvement also partially address 			this difficulty. \\ \bottomrule
\end{tabular}%
}
\end{table}