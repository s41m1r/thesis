\subsection{Related Work}
Role discovery has been addressed by literature in different settings and from several points of view. Here we highlight existing efforts from a data perspective.

\subsubsection{Structured data approaches}
This class of methods includes algorithms that make use of quantifiable data. We divide them into: 
\begin{inparaenum}[\itshape a)]
   \item \gls*{msr} approaches; and
   \item \gls*{pm} approaches.
\end{inparaenum}

\paragraph{\bfseries Mining software repositories approaches}
In the area of \gls*{msr}, \cite{Yu.LiguoRamaswamy.2007} use a hierarchical clustering based on user interactions to identify two categories of users: \emph{core member} and \emph{associate member}. Core members are those users whose interaction frequency is higher than a given threshold. Associate members are instead users whose interaction frequency is below the threshold.
\cite{Alonso2008} use a rule-based classifier that maps file types onto categories and hence each author who modified a file is linked to the files' category. ~\cite{gousios2008measuring} classify developers contribution based on \gls*{loc} changes and infer activities from them. \cite{Begel2010} developed the Codebook software tool a utility for finding experts. They use a social network approach that combines sources from people, artifacts, and textual references to other people. 
%Codebook is able to crawl several repositories and store the data as a graph in a database. Using a social network approach, that combines sources from people, artifacts, and as well as textual allusions to other people, this tool allows for finding experts. 
\cite{Ying2014} study developer profiles in terms of their interaction with the software artifacts to understand how they modify files and to further recommend changes based on history from \gls*{vcs} logs. \cite{Fuller2014a} investigate user roles in innovation-contest communities. They use quantitative methods to analyze user activity logs and interpretative to categorize qualitative comments into classes. 
%Furthermore, they used clustering to identify six user types: master, idea generator, socializer, efficient contributor, passive idea generator, passive commentator. Their work point out that different users are indeed characterized by behavioral contribution patterns. 

\paragraph{\bfseries Process mining approaches}
Efforts have been done to analyze software repositories with process mining techniques. \cite{rubin2007process} implement a multi-perspective incremental mining that is able to continuously integrate sources of evidence and improve the software engineering process as the user interacts with the documents in the repository. Their approach allows for mining other perspectives, such as roles, by applying social network analysis. However, only statistical methods can be applied to their output, since it lacks the comments that are associated to file changes. In the same setting, \cite{DBLP:conf/csmr/PoncinSB11} developed \textsc{frasr}, a framework for analyzing software repositories. \textsc{frasr} can be used in order to transform \gls*{vcs} logs data into the XES~\citep{verbeek2010xes} data format that can be further analyzed with process mining tools like ProM\footnote{http://www.promtools.org/doku.php}. 
\cite{Song2008} focus on three types of organizational mining 
\begin{inparaenum}[\itshape i)]
   \item organizational model mining, 
   \item social network analysis, and 
   \item information flows between organizational entities.
\end{inparaenum} 
\cite{Schonig2015} propose a mining technique to discover resource-aware declarative processes.

\subsubsection{Unstructured data} \cite{Maalej2010} use \gls*{nlp} for automating descriptions of work sessions by analyzing developers' informal text notes about their tasks. Developers are then classified into two classes based on their behavior: developers who use problem information to refer to their current activity and developers who refer to task and requirements. \cite{Kouters2012} developed an identity merging algorithm based on Latent Semantic Analysis (LSA) to disambiguate user emails. \cite{Licorish2014} mined developer comments to understand their attitudes.

\subsubsection{Other related work} The term \emph{role mining} often points to role mining algorithms based on \gls*{rbac} systems. These algorithms take as input predefined roles that are given as a matrix, where each user is assigned to access permissions. A number of algorithms have been developed to mine roles from \gls*{rbac} systems alone \citep{Lu2015,frank2013role} or combining their data with process history logs, as in \citep{baumgrass2012deriving}. A survey of existing techniques and algorithms can be found in \citep{Mitra2016}. Our work is disjoint from this class of algorithms as \gls*{vcs} does not contain access control information. 
\cite{Bhattacharya2014} propose a contributor graph-based model. By constructing a source-based profile, and a bug-based profile, they are able to identify seven roles: \emph{patch tester}, \emph{assist}, \emph{triager}, \emph{bug analyst}, \emph{core developer}, \emph{bug fixer}, and \emph{patch-quality improver}. \cite{hoda2013self} use a \gls*{gt} approach to study agile teams. Their work distinguishes the roles of \emph{mentor}, \emph{coordinator}, \emph{translator}, \emph{champion}, \emph{promoter}, and \emph{terminator}.


This work builds upon existing literature in that it strives to get insights on organizational level like in \cite{rubin2007process} and \cite{Song2008}, but it takes into account unstructured data. Differently from the literature that works with unstructured data, we explicitly consider the problem of role discovery, i.e. label resources with roles. Lastly, this approach differs from \cite{Bhattacharya2014} and \cite{hoda2013self} since we further adopt \gls*{nlp} techniques.
