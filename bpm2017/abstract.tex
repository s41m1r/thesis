%Project managers are interested in gathering insights into the software development process of their company. These insights can uncover important issues regarding the work structure, such as hidden dependencies among artifacts or unbalanced assignment of resources to tasks.
%%, such as non descriptive project structure (e.g., lack of modularization), unknown resource effort and hidden dependencies of artifacts.
%Existing approaches in literature tackle these issues from different perspectives. However, to the best of our knowledge, there is no current approach which allows the manager to obtain an overall view on the project from the work history perspective. %This paper presents an approach that generates a visualization which combines information about the evolution, containments and dependencies of artifacts by mining logs from version control systems.
%This paper presents an approach to extract and represent the work process reflected in the files and the dependencies among these files by mining logs from version control systems.
%We evaluate our approach using development software projects from GitHub, showing its potential use for project management support.

The monitoring of project-oriented business processes is difficult because their state is fragmented and represented by the progress of different documents and artifacts being worked on. This observation holds in particular for software development projects in which various developers work on different parts of the software concurrently. Prior contributions in this area have proposed a plethora of techniques to analyze and visualize the current state of the software artifact as a product. It is surprising that these techniques are missing to provide insights into what types of work are conducted at different stages of the project and how they are dependent upon another. In this paper, we address this research gap and present a technique for mining the software process including dependencies between artifacts. Our evaluation of various open-source projects demonstrates the applicability of our technique. 