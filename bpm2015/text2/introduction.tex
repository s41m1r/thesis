\section{Introduction}
Business process management plays an important role for improving the performance and compliance of various types of processes. In practice, many processes are executed with clear guidelines and regulatory rules, but without an explicit centralized control imposed by a process engine. In particular, it is often important to exactly know when which work was done. This is, for instance, the case for complex engineering processes in which different parties are involved. We refer to this class of processes as project-oriented business processes.

Such project-oriented business processes are difficult to control due to the lack of a centralized process engine. However, there are various unstructured pieces of information available to analyze and monitor their progress. One type of data that are often available these processes is event data from version control systems (VCS). While process mining techniques provide a useful perspective on how such event data can be analyzed, they do not produce output that is readily organized according to the project orientation of these processes.

In this paper, we define formal concepts for capturing project-oriented processes. These concepts provide the foundation for us to develop an automatic discovery technique which we refer to as \emph{project mining}. The output of our project mining algorithm is organized according to the specific structure typically encountered in project-oriented business processes. With this work, we extend the field of process mining towards the coverage of this specific type of business process.

The paper is structured as follows. Section~\ref{sec:background} describes the research problem and summarizes insights from prior research upon which our project mining approach is built. Section~\ref{sec:concept} defines the preliminaries of our work and presents an algorithm to mine project-oriented business processes. Section~\ref{sec:evaluation} describes the implementation of this algorithm and discusses the results from its application to VCS logs from a real-world engineering project. Section~\ref{sec:discuss} highlights the implications of this work before Section~\ref{sec:outro} concludes. 