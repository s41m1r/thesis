%\begin{abstract}
%Software development processes are complex to monitor as they involve the coordination of many resources working with different tools.
%This makes it hard to apply mining techniques for monitoring the process. 
%A key challenge for using traces of tools such as version control systems (VCS) is to find meaningful abstractions in order to identify the work that was actually done. 
%In this paper, we use data from VCS to analyze the actual progress of software-development processes. We develop a technique that is able to mine the activity types of which the development processes consists. 
%We implement our technique as a prototype in Java and evaluate its outputs in terms of effectiveness.
%In this way, we are able to graphically uncover new behavioural patterns in real-world data from existing open-source GitHub repositories.
%
%\begin{comment}
%The development of a software comprises different phases, e.g development, testing or documenting. A project manager must monitor the software development process to guarantee for instance that the expected development phase is the actual one being performed by the resources. However, software development processes are complex to monitor as they involve the coordination of a multitude of resources working with different tools. On the other hand, version control systems (VCS) as Git store in logs the work performed. An analysis of these logs can bring insights of how the software development process is progressing. In this paper, we develop a technique that mine data from VCS to discover the actual phase of a software-development processes. We implemented our technique as a prototype in Java. Along with the development phase our tool also outputs some Key Performance Indicators (KPIs). Its graphical interface allows the uncover of new patterns in real-world data from existing open-source GitHub repositories. We evaluate our technique using X of these repositories. 
%\end{comment}
%
%\keywords{activity discovery \and fine-grained event data \and mining software}
%\end{abstract}

