\section{Introduction}


Software development processes involve the coordination of multiple resources working on different parts of the software at the same time. As these resources focus on specific parts of the overall development process, it is difficult to obtain  transparency of the current status of the project. At the same time, monitoring such process is required in order to avoid risks of running out of time, budget or not meeting established quality objectives.

These type of processes have been referred to as \emph{project-oriented} business processes or simply as \emph{projects}. Monitoring a project is complex because there is hardly any central control of the work progress. 
% However, there is a myriad of trace data from different systems that is generated every day. One fundamental system in modern software projects are \glspl{vcs}. 
One type of system that is extensively used in software development are \glspl{vcs}.
They are used to keep track of the changes of the different files that constitute a project. Trace data generated by \gls{vcs} is a starting point for mining these type of processes. However, only a few approaches~\cite{DBLP:conf/bpm/BalaCMRP,DBLP:conf/bpm/JookenCJ19}  focus on analyzing the status of software development from these fine-grained \gls{vcs} traces. Thus, there is a need to fill in this gap with further techniques that allow to identify additional aspects of the development process, such as the actual activities done by developers. 

% On the other hand, there is a myriad of trace data from different systems that is generated every day. One fundamental system in modern software projects are \glspl{vcs}. These systems are used to keep track of the changes of the different files that constitute a project. Trace data generated by these systems is a starting point for mining software development processes. So far, only a few approaches exist (such as~\cite{DBLP:conf/bpm/BalaCMRP,DBLP:conf/bpm/JookenCJ19}) that help analyzing the status of software development. Support for understanding which types of work where conducted is missing.

In this paper, we provide a technique for capturing the progress of a project in such a way that it becomes clear what work activity is being done over time. We define fundamental concepts for representing these processes, upon which we develop a novel analysis technique. Our prototypical implementation of this technique is able to represent the status of the development process as well as the activity that is being done. 
%Kate: [2] was not explained so it is not clear the extension. This works advances the state-of-the art on project mining by extending the information offered by the technique developed by~\cite{DBLP:conf/bpm/BalaCMRP}.

The rest of the paper is organized as follows. \Cref{sec:background} details the problem, positions our contribution against existing literature, and defines the requirements for the design of the artifact that solves the stated problem. 
\Cref{sec:approach} defines preliminary concepts and presents the approach to mine the activities. %Kate: you describes the implementation in the section before 
\Cref{sec:results} describes the implementation of the artifact and shows its application to real-world projects. 
% \Cref{sec:discussion} highlights the implications of our results for project managers. 
\Cref{sec:conclusion} concludes the paper.