%\section{Future Work}
%the preliminary stage on which we plan to build upon a more comprehensive solution of the problem of role discovery from \gls*{vcs}. 
%There are many ways to extend our approaches as well as other potential methodologies for classification. 
%On the technical level, different methods of clustering and classification from the machine learning domain could be used. More conceptual additions would be for example the use of additional features and feature combinations.

%In this research we focused mainly on the user roles for classification but there might be other classes, for example differentiated by experience of users. 
%We expect improvement potential by analyzing different categories of VCS repositories. Especially log files of large software companies, where many different roles are formally defined and executed would be interesting.

%There is also much potential for further refining our proposed methods. For instance, we would use more extensively the natural language toolkit for processing the commit messages. So far we only used the more basic functionality of the language processing tools. In this research we focused mainly on the user roles for classification but there might be other classes, for example discriminated by the experience of users. Significant improvements would be possible by analyzing different categories of VCS repositories. Especially log files of large software companies, where many different roles are formally defined and executed would be interesting.


We presented a first approach to resource classification from \gls*{vcs} and demonstrated feasibility of this non-trivial task. The data saved in log files contains much useful information. But in particular the structure of the underlying team and project has a big influence on finding classes and their identified attributes. Further, the many differences between individual repositories make it difficult to generalize findings.

Automating the classification with machine learning is a viable method. Even though not all required steps of creating the classification models can be done automatically and further information about the users and structure is essential. Despite current challenges, the users can be classified to a certain extent. A more precise classification, however, requires future research. One direction is to further explore clustering and classification techniques to find better suited techniques. Finally, an extension of the approach with state-of-the-art semantic reasoning seems a promising vein of research to the authors.