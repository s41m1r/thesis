\section{The Conversational Structure of Problem-Solution Co-Evolution}
\label{sec:conversational-structure}

\todo[inline]{Do not focus too much on pull requests. This can be your evaluation scenario. Talk rather about people working with one another and trying to understand how their collaboration unfolds.

Maybe Aron and Damjan could be put together. This is because we want to gather information about the people.

``Social analyses''
}

\subsection{First section}

Open source software (OSS) development has become an important domain for studies who seek to explain the dynamics of new ways of organizing. Especially, free libre and open source software projects have been brought as an example of innovative ways of self-organizing~\cite{DBLP:journals/mansci/KroghH06,DBLP:journals/misq/HowisonC14}. Recent literature has emphasized iterations \cite{Berente2007}, self-organization and free-speech. But are they really that important?
%2. What is the overall problem or situation in that domain

Initiatives like the free software movement have often been understood as 
`taking freedom` in participating to open source projects, studying its underlying ideas, changing the code, redistributing the code to neighbors, who are allowed the same freedom. In other words, open source software development has been seen as a Libertarian practice (i.e., emphasizing individualism, freedom of choice and voluntary association). 

% The word "free" in our name does not refer to price; it refers to freedom. First, the freedom to copy a program and redistribute it to your neighbors, so that they can use it as well as you. Second, the freedom to change a program, so that you can control it instead of it controlling you; for this, the source code must be made available to you.

% Open Source Software development 
% Libertarian process (reference Aron \& other guys)
% but there is also some sort of Control!

%P2
%3. What are the conclusions of existing literature for that problem / situation in this domain? 
%4. What is the problem or issue with that existing literature?
However, it has been shown that control mechanisms (be there social or institutional) are strong in the area ~\cite{Lindberg2016}. Indeed, online collaborative work that generates knowledge goes through "negotiation" actions which “pull” the trajectory and shape its movement in the feature space~\cite{Arazy2020}. 

Q:
So, after all, is open source software development more institutionalized than corporate development?

%P3
% 5. Indicate that this study addresses that problem or issue and state how.
% 6. Describe the study, sample, and method for addressing that problem or issue.
% 7. Describe what you found.
% 8. State explicitly how these findings extend and contribute to existing knowledge. 

This study addresses the problem of freedom in open source development by comparing it to corporate bureaucracy. We analyze pull requests from a real world open source repository using speech act theory. This, in combination with data analysis techniques and process mining, allows us to untangle interesting insights about OSS development and point out relevant resemblance to stage-gate processes. 
%P4
%9. Describe the overall outline of the paper.

The rest of the paper is organized as follows. \Cref{sec:oss} describes the main aspects of OSS. \Cref{sec:short-vs-long-pr} describes the different pull request mechanisms. 