\chapter{Further Perspectives for Monitoring the Software Process}

% Management/Monitoring Perspective and How these techniques contribute
% Trace data 

% 
%
%
%Make the title shorter.
%
%\todo[inline]{Polish the chapter}
%\todo[inline]{Decide with Jan what to do here. Keep/Kill?}
%\todo[inline]{Elaborate more on the text. Motivate this section by including the social science point of view. The previous chapters are belong more to software engineering.}

The techniques presented in this dissertation fall under the broader category of analyses methods based on trace data. When used individually they provide specific insights about determinate perspective of a process. These methods can also be combined with one another in order to achieve a more complete picture about how the process performs under the different perspectives. Beyond measuring performance of processes, trace-data methods can be exploited in combination with other methods to aid theory development~\citep{DBLP:journals/isr/BerenteSS19}.

This chapter presents two more approaches that leverage trace data to gather insights on the software development. \Cref{sec:combining-questionnaier-and-trace} explores a mixed-method approach to identify relevant software development elements. It combines a subjective perspective from questionnaires and data analyses from event logs. As a result, insights are gathered on what elements both considered important and are actually used. \Cref{sec:conversational-structure} presents an approach that leverages speech act theory to understand how conversation about software development unfold. As a result, it is possible to observe differences in the process of successful collaboration. 

\chapter{Software development process evaluation based on integration of stakeholder perceptions and software development tool logs data}

\todo[inline]{Damjan work}

New important theoretical development for SDM evaluation - complementing stakeholder perceptions with data about actual execution of the software development process.

\section{What are the perceptions of developers and managers about the work?}

\section{What has been done in the past?}

\section{What we propose}

\section{Evaluation}

\section{Discussion}


\chapter{The Conversational Structure of Problem-Solution Co-Evolution}

\section{First section}

