\Extrachap{Abstract}


Software development is a highly flexible endeavor. Yet, software development projects
have specific requirements about quality, costs and time to deliver. Therefore, there is need
for monitoring and controlling the software development process. This problem is tackled
separately by the disciplines of software engineering and business process management. In the software engineering area, especially under the mining software repositories conference series, the focus is to develop new techniques and algorithms that exploit data science, artificial intelligence and machine learning to gather new information that helps to improve software engineering practices. In the business process management area, especially through process mining techniques, the focus is to gather insights about the exiting process of an organization with the goal of improving such process. While both disciplines focus on improvement, there are currently not many works that exploit this synergy.
On the one hand, software engineering rarely adopts a process point of view. Hence, they still fail at delivering knowledge about software development that can be easily understood by managers. On the other hand, process mining approaches from business process management, cannot be readily applied to trace data from software repositories. Therefore, they still fail at including trace data about these types of processes. 
 
This dissertation develops novel methods and algorithms that can be used to extract process knowledge from software development trace data. This knowledge is provided under four perspectives, namely time, case, organizational, and control-flow. Equipped with such multi-perspective knowledge of the process, a manager can obtain factual insights into projects. This dissertation can be positioned as a bridge between the process mining discipline and the area of software engineering. From a research point of view, the outcomes of this dissertation are a building block to further cross-fertilize research streams on process mining and software engineering. From a practical
point of view, the outcomes of this dissertation provides familiar means of analysis on software development to the project manager who wants to gather insights on the underlying process.


%\todo[inline]{KR: Confusing. It is not possible to understand the problem and your solution is also not clear. 
%	
%	You start with software development and then process appears. Later you bring process mining, but it is not clear why. Also, you talk about trace out of the blue. Maybe you can start from process of software developmentt}

