\chapter{Uncovering the Hidden Co-Evolution in the Work History of Software Projects}

\todo[inline]{rephrase}

\abstract*{The monitoring of project-oriented business processes is difficult because their state is fragmented and represented by the progress
	of different documents and artifacts being worked on. This observation
	holds in particular for software development projects in which various
	developers work on different parts of the software concurrently. Prior contributions in this area have proposed a plethora of techniques to analyze
	and visualize the current state of the software artifact as a product. It is
	surprising that these techniques are missing to provide insights into what
	types of work are conducted at different stages of the project and how
	they are dependent upon another. In this paper, we address this research
	gap and present a technique for mining the software process including
	dependencies between artifacts. Our evaluation of various open-source
	projects demonstrates the applicability of our technique.}

\abstract{The monitoring of project-oriented business processes is difficult because their state is fragmented and represented by the progress
	of different documents and artifacts being worked on. This observation
	holds in particular for software development projects in which various
	developers work on different parts of the software concurrently. Prior contributions in this area have proposed a plethora of techniques to analyze
	and visualize the current state of the software artifact as a product. It is
	surprising that these techniques are missing to provide insights into what
	types of work are conducted at different stages of the project and how
	they are dependent upon another. In this paper, we address this research
	gap and present a technique for mining the software process including
	dependencies between artifacts. Our evaluation of various open-source
	projects demonstrates the applicability of our technique.}


\section{Monitoring the Work History of Software Projects}

\todo[inline]{rephrase}

In this paper, we focus on a specific class of project-oriented business processes,
namely software development processes. These processes share some common
characteristics. First, they involve various resources with different roles. In the
simplest case, we can distinguish project managers and project participants.
Project managers are responsible for managing the development process and
supervising the work of the project participants, who in turn are responsible for
specific work tasks. Second, such processes are usually subject to constraints in
terms of cost, time and quality, which is mostly associated with the performance
of each of the work tasks. Third, the project participants work on a plethora of
artifacts, which are logically organized in a hierarchical structure, with complex
interdependencies among them. Given these characteristics, it is the goal of theproject manager to organize the software development process in such a way
that the work on different files and tasks reflects the complex interdependencies,
the constraints and the available participants. Therefore, it is important for the
manager to understand the work history of the process in order to monitor the
progress systematically.

\section{Approaches to Monitoring Software Projects}

\todo[inline]{What has the state of the art done.}


