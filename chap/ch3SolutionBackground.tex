%\chapter{Analysis Techniques for Software Processes}
%\label{chap:ch3-solution-background}
%
%\todo[inline]{This is still related work, but here we talk about the "solutions". Check for overlap with Research Background and the other chapters below.}
%\todo[inline]{Elaborate on Section 3.2, 3.3}


%This chapter is concerned with data-driven approaches used in literature to \emph{solutions}, which help tackling the problem of monitoring software development. Four disciplines provide the necessary theoretical and practical knowledge upon which this thesis is built. Thus, this chapter presents these four disciplines as follows. \Cref{sec:process-mining} briefly describes the area of process mining. \Cref{sec:msr} describes contributions from the area of mining software repositories. \Cref{sec:visualization} provides an overview on the area of software visualization. \Cref{sec:text-mining} shortly describes the area of text mining, used for extracting knowledge from unstructured data. Finally, \Cref{sec:summary-of-relevant-techniques} summarizes these solutions. 



%\section{Text Mining}
%\label{sec:text-mining}
%
%%\subsection{Description of the field} 
%
%\Gls{tm} refers to the process of deriving information from natural language text. It relates to data mining in the aspect that both strive to extract meaningful information from raw data. However, data mining is characterized as the extraction of implicit, previously unknown, and potentially useful information from data, whereas with text mining the information to be extracted is clearly and explicitly stated in the text~\citep{Witten2004}. 
%\Gls{tm} can be regarded as going beyond information access to further help users analyze and digest information and facilitate decision making. There are also many applications of text mining where the primary goal is to analyze and discover any interesting patterns, including trends and outliers in text data.
%Although text mining is mostly about \gls{nlp}~\citep{jurafsky2014speech}, it embraces also applications that go beyond. For example, it analyzes linkage structures such as the citations in the academic literature and hyperlinks in the Web literature, both useful sources of information that lie outside the traditional domain of \gls{nlp}. 

%The main types of text mining are the following.
%
%%\Cref{tab:text-mining-overview} gives a compact overview on the types of text mining and some example references.
%%
%%\begin{table}[!h]
\centering
\caption{Overview on text mining types}
\label{tab:text-mining-overview}
\vspace{5pt}
%\resizebox{\textwidth}{!}{
\begin{tabular}{m{4cm}m{5cm}m{4cm}}
\toprule
\textbf{Type}                         & \textbf{Task}                                                                                                                   & \textbf{References}                                                                                           \\ \midrule
Information extraction       & Filling in templates from natural language text                                                                        & \cite{cowie1996information}, \cite{mooney1999relational}, \cite{seymore1999learning}, \cite{banko2007open} \\ %\midrule
Topic detection and tracking & Finding and following new events in a stream                                                                           & \cite{Allan1998}, \cite{wayne2000multilingual}                                                  \\ %\midrule
Summarization                & Reducing the content obtained from text documents, still keeping the topic                                             & \cite{aggarwal2012mining}, \cite{gupta2009survey}                                                    \\ %\midrule
Categorization               & Identifying the main themes of a document                                                                              & \cite{sebastiani2002machine}, \cite{joachims1998text}                                                \\ %\midrule
Clustering                   & Group similar documents by predefined topics                                                                           & \cite{zhao2001criterion}, \cite{fung2003hierarchical}, \cite{aggarwal2012mining}                     \\ %\midrule
Concept Linkage              & Connect related documents by identifying their shared concepts                                                         & \cite{maedche2000mining}, \cite{fan2006tapping}, \cite{gupta2009survey}                             \\ %\midrule
Information visualization    & Visualizing large textual sources                                                                                      & \cite{wong1999visualizing}, \cite{Mostafa2013}                                                       \\ %\midrule
Question answering           & Automatically answer questions posed by humans                                                                         & \cite{katz1997sentence}, \cite{kwok2001scaling}, \cite{aggarwal2012mining}                           \\ %\midrule
Association rule mining      & Study the relationships and implications among topics or descriptive concepts that are used to characterize a corpus & \cite{agrawal1993mining}, \cite{agrawal1994fast}, \cite{hu2010}                                      \\ \bottomrule
\end{tabular}
%}
\end{table}
%
%\begin{description}
%	\item[Information extraction.] 
%%	Information extraction is used to refer to the task of filling in templates from natural language text. The goal is to extract from the documents (which may be in a variety of languages) salient facts about prespecified types of events, entities or relationships. These facts are then usually entered automatically into a database, which may then be used to analyze the data for trends, to give a natural language summary, or simply to serve for on-line access. Traditional information extraction techniques~\citep{cowie1996information,mooney1999relational} leverage on rule-based systems that match predefined linguistic patterns. More recently, work on named entity recognition uses statistical machine learning methods~\citep{seymore1999learning}. A tool that uses unsupervised learning can be found in \citep{banko2007open}.
%	It is the task of filling in templates from natural language text. The goal is to extract from the documents (which may be in a variety of languages) salient facts about prespecified types of events, entities or relationships. They can be used to analyze the data for trends, to give a natural language summary, or simply to serve for on-line access. Typical examples can be found in~\citep{cowie1996information,mooney1999relational,seymore1999learning,banko2007open}.
%	
%	\item[Topic detection and tracking.] 
%%	\Gls{tdt} was a DARPA-sponsored initiative to investigate on finding and following new events in a stream of broadcast news stories. The \gls{tdt} problem consists of three major tasks: (1) \emph{segmenting} a stream of data, especially recognized speech, into distinct stories; (2) \emph{identifying} those news stories that are the first to discuss a new event occurring in the news; and (3) given a small number of sample news stories about an event, \emph{finding} all \emph{following} stories in the stream.
%%	The work of \citep{allan2002introduction} has formally defined this problem and proposed the initial set of algorithms for the task. Main subtasks of TDT, as identified in \citep{wayne2000multilingual} are 
%%	(i) finding topically homogeneous regions (segmentation); (ii) finding additional stories about a given topic (tracking); (iii) detecting and threading together new topics (detection); (iv) Detecting new topics (first story detection); and (v) Deciding whether stories are on the same topic (linking). An example of a real-world TDT system is Google Alerts\footnote{\url{https://www.google.com/alerts}}.
%	The \gls{tdt} problem consists of three major tasks: (1) \emph{segmenting} a stream of data, especially recognized speech, into distinct stories; (2) \emph{identifying} those news stories that are the first to discuss a new event occurring in the news; and (3) given a small number of sample news stories about an event, \emph{finding} all \emph{following} stories in the stream.
%	Typical examples can be found in~\citep{wayne2000multilingual,allan2002introduction}.
%	%	The work of \citep{allan2002introduction} has formally defined this problem and proposed the initial set of algorithms for the task. Main subtasks of TDT, as identified in \citep{wayne2000multilingual} are 
%	%	(i) finding topically homogeneous regions (segmentation); (ii) finding additional stories about a given topic (tracking); (iii) detecting and threading together new topics (detection); (iv) Detecting new topics (first story detection); and (v) Deciding whether stories are on the same topic (linking). An example of a real-world TDT system is Google Alerts\footnote{\url{https://www.google.com/alerts}}.
%	
%	\item[Summarization.] 
%%	Summarization is the task of reducing the content obtained from text documents, still keeping a brief overview on a topic that they treat. Summarization techniques generally fall into two categories \citep{Aggarwal2015}. In extractive summarization, a summary consists of information units extracted from the original text; in contrast, in abstractive summarization, a summary may contain {\textquotedblleft synthesized\textquotedblright} information units that may not necessarily occur in the text document. An automatic summarization process can be divided into three steps \citep{gupta2009survey}: (1) the \emph{preprocessing} step where a structured representation of the original text is obtained; (2) the \emph{processing} step where an algorithm must transform the text structure into a summary structure; and (3) the \emph{generation} step where the final summary is obtained from the summary structure. 
%%	A plethora of text summarization tools can be found online. A few examples are the open source libraries such Open Text Summarizer\footnote{\url{https://www.splitbrain.org/services/ots}}, Sumplify\footnote{\url{http://sumplify.com/}} and Online summarize tool\footnote{\url{http://www.tools4noobs.com/summarize/}}.
%Summarization is the task of reducing the content obtained from text documents, still keeping a brief overview on a topic that they treat. Typical examples of can be found in~\citep{gupta2009survey,Aggarwal2015}. Some open source libraries are Open Text Summarizer\footnote{\url{https://www.splitbrain.org/services/ots}}, Sumplify\footnote{\url{http://sumplify.com/}} and Online summarize tool\footnote{\url{http://www.tools4noobs.com/summarize/}}.
%	
%	\item[Categorization.] 
%%	Categorization aims at identifying the main themes of a document. This translates into assigning natural language documents to predefined categories according to their content~\citep{sebastiani2002machine}. Categorization often relies on a thesaurus for which topics are predefined, and relationships are identified by looking for broad terms, narrower terms, synonyms, and related terms. \Glspl{svm} are used to automatically learn text classifiers from examples~\citep{joachims1998text}. Categorization tools usually rank documents according to how much of their content fits in a particular topic. 
%%	
%	Categorization aims at assigning natural language documents to predefined categories according to their content. 
%%	Categorization often relies on a thesaurus for which topics are predefined, and relationships are identified by looking for broad terms, narrower terms, synonyms, and related terms. 
%	\Glspl{svm} are used to automatically learn text classifiers from examples. Typical examples can be found in~\citep{joachims1998text,sebastiani2002machine}.
%%	Categorization tools usually rank documents according to how much of their content fits in a particular topic. 
%	
%	\item[Clustering.] 
%%	Clustering is a technique used to group similar documents, but it differs from categorization in that documents are clustered on the fly instead of through predefined topics. Documents can also appear in multiple subtopics, ensuring that useful documents are not omitted from the search results. A basic clustering algorithm creates a vector of topics for each document and measures the weights of how the document fits into each cluster. A survey of clustering algorithms can be found in \citep{Aggarwal2015}. 
%	Clustering is a technique used to group similar documents, but it differs from categorization in that documents are clustered on the fly instead of through predefined topics. 
%%	Documents can also appear in multiple subtopics, ensuring that useful documents are not omitted from the search results. A basic clustering algorithm creates a vector of topics for each document and measures the weights of how the document fits into each cluster. 
%A literature review of clustering algorithms can be found in \citep{Aggarwal2015}. 
%	
%	\item[Concept Linkage.] 
%%	Concept-linkage tools connect related documents by identifying their shared concepts, helping users find information they perhaps would not have found through traditional search methods~\citep{gupta2009survey}. It promotes browsing for information rather than searching for it. For example, a text mining software solution may easily identify a transitive closure in a set of topics \{X, Y, Z\}, i.e., a link between X and Y, a link between Y and Z, %. But the text mining tool could also detect a potential 	and a link between X and Z. %(i.e. transitive closure on the set of topics), something that a human researcher has not come across yet.	With large sets of data, such a relationship could be disregarded by humans. %because of the large volume of information s/he would have to sort through to make the connection.
%	Concept-linkage tools connect related documents by ide\section{Text Mining}
%\label{sec:text-mining}
%
%%\subsection{Description of the field} 
%
%\Gls{tm} refers to the process of deriving information from natural language text. It relates to data mining in the aspect that both strive to extract meaningful information from raw data. However, data mining is characterized as the extraction of implicit, previously unknown, and potentially useful information from data, whereas with text mining the information to be extracted is clearly and explicitly stated in the text~\citep{Witten2004}. 
%\Gls{tm} can be regarded as going beyond information access to further help users analyze and digest information and facilitate decision making. There are also many applications of text mining where the primary goal is to analyze and discover any interesting patterns, including trends and outliers in text data.
%Although text mining is mostly about \gls{nlp}~\citep{jurafsky2014speech}, it embraces also applications that go beyond. For example, it analyzes linkage structures such as the citations in the academic literature and hyperlinks in the Web literature, both useful sources of information that lie outside the traditional domain of \gls{nlp}. 
%ntifying their shared concepts, helping users find information they perhaps would not have found through traditional search methods. Typical examples can be found in~\citep{gupta2009survey}. 
%%	It promotes browsing for information rather than searching for it. For example, a text mining software solution may easily identify a transitive closure in a set of topics \{X, Y, Z\}, i.e., a link between X and Y, a link between Y and Z, %. But the text mining tool could also detect a potential 
%%	and a link between X and Z. %(i.e. transitive closure on the set of topics), something that a human researcher has not come across yet 
%%	With large sets of data, such a relationship could be disregarded by humans. %because of the large volume of information s/he would have to sort through to make the connection.
%	
%	\item[Information Visualization.] 
%%	Information visualization aims at visualizing large textual sources in such a way that the content can be displayed in a hierarchy or map and provides browsing feat\section{Text Mining}
%\label{sec:text-mining}
%
%%\subsection{Description of the field} 
%
%\Gls{tm} refers to the process of deriving information from natural language text. It relates to data mining in the aspect that both strive to extract meaningful information from raw data. However, data mining is characterized as the extraction of implicit, previously unknown, and potentially useful information from data, whereas with text mining the information to be extracted is clearly and explicitly stated in the text~\citep{Witten2004}. 
%\Gls{tm} can be regarded as going beyond information access to further help users analyze and digest information and facilitate decision making. There are also many applications of text mining where the primary goal is to analyze and discover any interesting patterns, including trends and outliers in text data.
%Although text mining is mostly about \gls{nlp}~\citep{jurafsky2014speech}, it embraces also applications that go beyond. For example, it analyzes linkage structures such as the citations in the academic literature and hyperlinks in the Web literature, both useful sources of information that lie outside the traditional domain of \gls{nlp}. 
%ures, in addition to simple search. For instance, governments or police can identify terrorist networks in a map and identify crimes that were previously unconnected~\citep{gupta2009survey}.
%	Information visualization aims at visualizing large textual sources in such a way that the content can be displayed in a hierarchy or map and provides browsing features, in addition to simple search. Typical examples can be found in~\citep{gupta2009survey}.
%	
%	\item[Question Answering.] 
%%	Another application area of developed text-mining technologies, along with the natural language processing, is natural language features Question Answering (QA). QA is concerned with building systems that automatically answer questions posed by humans in a natural language.
%%	One of the first Query Answering tools was START\footnote{\url{http://start.csail.mit.edu/index.php}}, developed in the work of \citep{katz1997sentence}. Question answering has been extensively used in Biomedical domain for aiding researchers and health care professionals in managing the continuous growth of information.	
%%	Another application area of developed text-mining technologies, along with the natural language processing, is natural language features Question Answering (QA). 
%	QA is concerned with building systems that automatically answer questions posed by humans in a natural language. One of the first Query Answering tools was START\footnote{\url{http://start.csail.mit.edu/index.php}}, developed in the work of \citep{katz1997sentence}. 
%%	Question answering has been extensively used in Biomedical domain for aiding researchers and health care professionals in managing the continuous growth of information.
%	
%	\item[Association Rule Mining.] 
%%	The focus of association rules mining is to study the relationships and implications among topics, or descriptive concepts, that are used to characterize a corpus. The work of \citep{agrawal1994fast} shows how it is possible to discover association rules from massive data from databases, referred to as \emph{basket} data. The same approach can be followed by constructing a database of rules using information extraction methods and subsequently applying techniques, e.g. \citep{hu2010}, to uncover hidden associations in the database. For instance, a rule might be that 98\% of customers that purchase tires and auto accessories also get automotive services done. This suggests that association rules can be captured as if/then patterns. Criteria such as support and confidence~\citep{agrawal1993mining} are used to identify the most important relationships. Support is an indication of how frequently the items appear in the database. Confidence indicates the number of times the if/then statements has been found to be true.
%	The focus of association rules mining is to study the relationships and implications among topics, or descriptive concepts, that are used to characterize a corpus. Typical examples can be found in~\citep{agrawal1993mining,agrawal1994fast,hu2010}.
%%	The work of \citep{agrawal1994fast} shows how it is possible to discover association rules from massive data from databases, referred to as \emph{basket} data. The same approach can be followed by constructing a database of rules using information extraction methods and subsequently applying techniques, e.g. \citep{hu2010}, to uncover hidden associations in the database. For instance, a rule might be that 98\% of customers that purchase tires and auto accessories also get automotive services done. This suggests that association rules can be captured as if/then patterns. Criteria such as support and confidence~\citep{agrawal1993mining} are used to identify the most important relationships. Support is an indication of how frequently the items appear in the database. Confidence indicates the number of times the if/then statements has been found to be true.
%	
%\end{description}

%\subsection{Contribution to the research questions} 
%
%%\todoinline{more details}
%
%%My research involves unstructured data in the form of word processor documents, emails, user forums, and commit messages from \gls{vcs}. Text mining algorithms can be combined to quantitative data to gather information about project work. For example, topics models can be combined to time-clustered messages from pull request comments. In this way it is possible order to discover ``hot topics" (e.g. deliverable, milestone, meeting, etc) during project development. More in general, text mining techniques help with preprocessing the data and retrieving relevant information. Successively, this information can be brought together to create a structured event log in the \gls{xes} format. Therefore, text mining research serves and a preprocessing technique to facilitate process extraction. Especially, it helps as follows. 
%
%\gls{nlp} based approaches have been used to identify process cases and activities. Contributions in this group the have focused on process discovery. \citep{Goncalves2009a} discovered a model from group stories. \citep{Friedrich2011} reaches 77\% of accuracy in reconstructing process models from text. \citep{DoNascimento2012} analyze legacy systems code to infer business process rules and activities. \citep{DiCiccio2013} use \gls{nlp} to aid the extraction of artful processes from knowledge workers emails.
%
%There are also works that help with information extraction. \citep{Maalej2010} use \gls{nlp} for automating descriptions of work sessions by analyzing developers' informal text notes about their tasks. Developers are then classified into two classes based on their behavior: developers who use problem information to refer to their current activity and developers who refer to task and requirements. \citep{Kouters2012} developed an identity merging algorithm based on Latent Semantic Analysis (LSA) to disambiguate user emails. \citep{Licorish2014} mined developer comments to understand their attitudes.
%
%Role discovery has also seen contributions. A number of algorithms have b\subsection{Contribution to the research questions} 

%\todoinline{more details}

%My research involves unstructured data in the form of word processor documents, emails, user forums, and commit messages from \gls{vcs}. Text mining algorithms can be combined to quantitative data to gather information about project work. For example, topics models can be combined to time-clustered messages from pull request comments. In this way it is possible order to discover ``hot topics" (e.g. deliverable, milestone, meeting, etc) during project development. More in general, text mining techniques help with preprocessing the data and retrieving relevant information. Successively, this information can be brought together to create a structured event log in the \gls{xes} format. Therefore, text mining research serves and a preprocessing technique to facilitate process extraction. Especially, it helps as follows. 

%\gls{nlp} based approaches have been used to identify process cases and activities. Contributions in this group the have focused on process discovery. \citep{Goncalves2009a} discovered a model from group stories. \citep{Friedrich2011} reaches 77\% of accuracy in reconstructing process models from text. \citep{DoNascimento2012} analyze legacy systems code to infer business process rules and activities. \citep{DiCiccio2013} use \gls{nlp} to aid the extraction of artful processes from knowledge workers emails.
%
%There are also works that help with information extraction. \citep{Maalej2010} use \gls{nlp} for automating descriptions of work sessions by analyzing developers' informal text notes about their tasks. Developers are then classified into two classes based on their behavior: developers who use problem information to refer to their current activity and developers who refer to task and requirements. \citep{Kouters2012} developed an identity merging algorithm based on Latent Semantic Analysis (LSA) to disambiguate user emails. \citep{Licorish2014} mined developer comments to understand their attitudes.
%
%Role discovery has also seen contributions. A number of algorithms have been developed to mine roles from \gls{rbac} systems alone (\citep{Lu2015,frank2013role}) or combining their data with process history logs, as in ~\citep{baumgrass2012deriving}. A survey of existing techniques and algorithms can be found in \citep{Mitra2016}.
%
%This body of works 
%%suggests that process insights can be obtained from unstructured data. \Gls{tm} supports my research with preprocessing to facilitate process extraction. Especially, it 
%helps with:
%\begin{inparaenum}[\itshape i)]
%	\item \textbf{RQ2} -- extracting case characteristics from conversations in user forums;
%	\item \textbf{RQ3} -- extracting resource roles from user comments in commit messages;
%	\item \textbf{RQ4} -- extracting activities out of coordination message exchanges.
%\end{inparaenum} 
%
%\subsection{Limitations}
%\Gls{tm} techniques do not natively support extraction of processes, but they can be used to help addressing information extraction from unstructured data. \Gls{tm} approaches focus on obtaining structured information from unstructured textual data, and mainly uses \gls{nlp}~\citep{Witten:1999}. Works that use \gls{nlp} can be found in both \gls{pm} \citep{VanderAalst2016b} and \gls{msr} \citep{Chen2016a}. In the \gls{bpm} \citep{Dumas2018} area, \gls{nlp} techniques have been used to understand process activities~\citep{Leopold2013,Mendling2014} and analyze software processes under a knowledge-intensive perspective~\citep{DeA.R.Goncalves2011,Richetti2017}. Likewise, in the~\gls{msr} area, \gls{nlp} has been used as an information extraction tool to obtain informative metrics from a software engineering perspective~\citep{Thomas2014,Chen2016a}. 
%
%\todo[inline]{CHECK! As with the previous section}
%
%Text mining techniques include state of the art NLP methods to extract knowledge from unstructured data. These methods exploit the communication channels used by developers. Such unstructured data is rich in information. 
%Relevant techniques focus on these data.
%Any technique is enriching the event log using these data?een developed to mine roles from \gls{rbac} systems alone (\citep{Lu2015,frank2013role}) or combining their data with process history logs, as in ~\citep{baumgrass2012deriving}. A survey of existing techniques and algorithms can be found in \citep{Mitra2016}.
%
%This body of works 
%%suggests that process insights can be obtained from unstructured data. \Gls{tm} supports my research with preprocessing to facilitate process extraction. Especially, it 
%helps with:
%\begin{inparaenum}[\itshape i)]
%	\item \textbf{RQ2} -- extracting case characteristics from conversations in user forums;
%	\item \textbf{RQ3} -- extracting resource roles from user comments in commit messages;
%	\item \textbf{RQ4} -- extracting activities out of coordination message exchanges.
%\end{inparaenum} 
%
%\subsection{Limitations}
%\Gls{tm} techniques do not natively support extraction of processes, but they can be used to help addressing information extraction from unstructured data. \Gls{tm} approaches focus on obtaining structured information from unstructured textual data, and mainly uses \gls{nlp}~\citep{Witten:1999}. Works that use \gls{nlp} can be found in both \gls{pm} \citep{VanderAalst2016b} and \gls{msr} \citep{Chen2016a}. In the \gls{bpm} \citep{Dumas2018} area, \gls{nlp} techniques have been used to understand process activities~\citep{Leopold2013,Mendling2014} and analyze software processes under a knowledge-intensive perspective~\citep{DeA.R.Goncalves2011,Richetti2017}. Likewise, in the~\gls{msr} area, \gls{nlp} has been used as an information extraction tool to obtain informative metrics from a software engineering perspective~\citep{Thomas2014,Chen2016a}. 
%
%\todo[inline]{CHECK! As with the previous section}
%
%Text mining techniques include state of the art NLP methods to extract knowledge from unstructured data. These methods exploit the communication channels used by developers. Such unstructured data is rich in information. 
%Relevant techniques focus on these data.
%Any technique is enriching the event log using these data?

\section{Summary}
\label{sec:ch2-summary}
\todo[inline]{Write summary}

%\label{sec:summary-of-relevant-techniques}
%
%\todo[inline]{Here go the techniques relevant for this thesis.} 
%
%This research combines ideas from the above mentioned areas to devise algorithms for mining the software development process. This section defines the contribution of related fields to the four research question. \Cref{table:literature-classification} lists summarizes the most relevant works in the literature and classifies them in relation the addressed research questions.
%
%% Please add the following required packages to your document preamble:
% \usepackage{booktabs}
\begin{table}[]
%\vspace*{-\baselineskip}
\centering
\caption{Classification of existing literature addressing the various aspects of mining the development process}
\label{table:literature-classification}
\begin{tabular}{@{}m{.7cm}>{\raggedright}m{3cm}>{\raggedright}m{2.5cm}>{\raggedright}m{2.5cm}>{\raggedright\arraybackslash}m{2.5cm}@{}}
\toprule
\multicolumn{1}{l}{} & \textbf{R1}  (\emph{time})                                             & \textbf{R2}  (\emph{case})                                                                                             & \textbf{R3} (\emph{organisation} )& \textbf{R4}   (\emph{control-flow})                                   \\ \midrule

{\rotatebox[origin=c]{90}{\textbf{Process Mining}}} & Dotted Chart \citep{Song2007} & Decision mining \citep{Rozinat2006} & Visualization techniques \citep{Baumgrass2013} Organizational mining \citep{Song2008} \citep{Schonig2015} & Bug fixing \citep{DBLP:conf/csmr/PoncinSB11}, Workflow fragments \citep{DBLP:conf/se/KindlerRS06,kindler2006incremental} \\ \cmidrule(lr){2-5}

{\rotatebox[origin=c]{90}{\textbf{This work}}} & Artifact \textbf{A1} & Artifact \textbf{A2} & Artifact \textbf{A3} & Artifact \textbf{A4} \\ \cmidrule(lr){2-5}

 
{\rotatebox[origin=c]{90}{\textbf{Mining Software Repositories}}} & Time series \citep{Ruohonen2015} \citep{Hou2014}, Statistical analyses \citep{Oliva2011}, Information extraction \citep{cowie1996information} & 

Network analysis \citep{DAmbros2009}, \citep{Zimmermann2008}, Language Models \citep{Allamanis2013},  Topic models \citep{Chen2016a} Theory-generating case studies \citep{Lindberg2016} 
& 

Email analysis \citep{Bird2006}, Role identification \citep{Yu.LiguoRamaswamy.2007} Social network \citep{Bird2006} Survey \citep{Begel2010} \citep{DeA.R.Goncalves2010}

& Exploratory studies \citep{Gousios2014} Speech acts \citep{DiCiccio2013a} \citep{Campos2018} Natural language processing \citep{Friedrich2011} \\ \bottomrule
\end{tabular}
\vspace*{-\baselineskip}
\end{table}

