%%%%%%%%%%%%%%%%%%%%% appendix.tex %%%%%%%%%%%%%%%%%%%%%%%%%%%%%%%%%
%
% sample appendix
%
% Use this file as a template for your own input.
%
%%%%%%%%%%%%%%%%%%%%%%%% Springer-Verlag %%%%%%%%%%%%%%%%%%%%%%%%%%

\appendix

%\motto{All's well that ends well}
\chapter{Additional artifacts}
\label{introA} % Always give a unique label
% use \chaptermark{}
% to alter or adjust the chapter heading in the running head

\section{Dataset from Industry Partner}
\label{sec:dataset-sdm}

% Please add the following required packages to your document preamble:
% \usepackage{booktabs}
% \usepackage{graphicx}
\begin{table}[h]
	\centering
	\caption{Analyzed sprints in dataset from Industry Partner.}
	\label{tab:sprints-counts}
	\resizebox{\textwidth}{!}{%
		\begin{tabular}{@{}llllll@{}}
			\toprule
			Sprint & Start Date & End Date   & No. features & No. implementation & No. bugs \\ \midrule
			1      & 2015-12-30 & 2016-01-12 & 9            & 0                  & 0        \\
			2      & 2016-01-13 & 2016-01-26 & 18           & 2                  & 1        \\
			3      & 2016-01-27 & 2016-02-09 & 2            & 0                  & 11       \\
			4      & 2016-02-10 & 2016-02-23 & 4            & 3                  & 9        \\
			5      & 2016-02-24 & 2016-03-08 & 5            & 3                  & 17       \\
			6      & 2016-03-09 & 2016-03-22 & 1            & 0                  & 2        \\
			7      & 2016-03-23 & 2016-04-05 & 8            & 5                  & 9        \\
			8      & 2016-04-06 & 2016-04-19 & 5            & 13                 & 12       \\
			9      & 2016-04-20 & 2016-05-03 & 5            & 5                  & 8        \\
			10     & 2016-05-04 & 2016-05-17 & 4            & 25                 & 19       \\
			11     & 2016-05-18 & 2016-05-31 & 4            & 2                  & 28       \\
			12     & 2016-06-01 & 2016-06-14 & 1            & 10                 & 12       \\
			13     & 2016-06-15 & 2016-06-28 & 0            & 0                  & 16       \\
			14     & 2016-06-29 & 2016-07-12 & 0            & 1                  & 10       \\
			15     & 2016-07-13 & 2016-07-26 & 0            & 6                  & 12       \\
			16     & 2016-07-27 & 2016-08-09 & 1            & 9                  & 10       \\
			17     & 2016-08-10 & 2016-08-23 & 2            & 6                  & 4        \\
			18     & 2016-08-24 & 2016-09-06 & 6            & 13                 & 12       \\
			19     & 2016-09-07 & 2016-09-20 & 2            & 6                  & 14       \\
			20     & 2016-09-21 & 2016-10-04 & 6            & 12                 & 12       \\
			21     & 2016-10-05 & 2016-10-18 & 3            & 13                 & 5        \\
			22     & 2016-10-19 & 2016-11-01 & 2            & 14                 & 1        \\
			23     & 2016-11-02 & 2016-11-15 & 0            & 5                  & 5        \\
			24     & 2016-11-16 & 2016-11-29 & 1            & 6                  & 10       \\
			25     & 2016-11-30 & 2016-12-13 & 4            & 9                  & 11       \\
			26     & 2016-12-14 & 2016-12-27 & 0            & 1                  & 12       \\
			27     & 2016-12-28 & 2017-01-10 & 1            & 4                  & 18       \\
			28     & 2017-01-11 & 2017-01-24 & 3            & 0                  & 14       \\
			29     & 2017-01-25 & 2017-02-07 & 1            & 5                  & 12       \\
			30     & 2017-02-08 & 2017-02-21 & 1            & 0                  & 14       \\
			31     & 2017-02-22 & 2017-03-07 & 0            & 0                  & 10       \\
			32     & 2017-03-22 & 2017-04-04 & 0            & 0                  & 8        \\
			33     & 2017-04-05 & 2017-04-18 & 1            & 0                  & 11       \\
			34     & 2017-04-19 & 2017-05-02 & 0            & 0                  & 9        \\
			35     & 2017-05-03 & 2017-05-16 & 4            & 0                  & 5        \\ \bottomrule
		\end{tabular}%
	}
\end{table}


\section{Guideline for Speech Act Annotation}
\label{sec:sa-guideline}

The frequent abbreviations used on GitHub (WDYT – What do you think?, IMO – In my opinion, ...)
should be labelled based on their meaning (Question, Opinion)
If a part of the comment is code ‘’’ code part ‘’’, it can not be any other speech act
If a part of the comment is a citation, it can not be any other speech act
Comments can have multiple speech acts. Each phrase can only have one speech act, but each
sentence can have multiple speech acts.
E.g.:
I will look at your code, @user123 thank you for asking. (This sentence would include 3 Speech Acts)
IMO this code looks difficult. (This sentence would only have one Speech Act)
For simplification reasons, categories consisting of more than one speech act (statement/description)
were named only as the first (statement).

We enter the according speech acts in the provided excel file, providing either a 0 (no occurrence
of this speech act) or the frequency a certain speech act appears in the comment.


% Please add the following required packages to your document preamble:
% \usepackage{booktabs}
% \usepackage{graphicx}
% \usepackage[table,xcdraw]{xcolor}
% If you use beamer only pass "xcolor=table" option, i.e. \documentclass[xcolor=table]{beamer}

