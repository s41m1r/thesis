\chapter{Conclusion}
\label{ch9-conclusion}

This chapter summarizes the main contributions of the dissertation. \Cref{sec:summary-of-results} lists the results. \Cref{sec:8-discussion} discusses in more detail their impact. \Cref{sec:8-future} outlines future research. 

\section{Summary of Results}
\label{sec:summary-of-results}

This doctoral thesis aims at bridging the gap between automatic analysis of software development data and process mining. It provides analyses techniques that help learning specific characteristics of the software process from its trace data. This thesis makes the following contributions.

\begin{itemize}
	
	\item \textbf{Conceptualization of project-oriented business processes.} Processes have been seen so far as repeatable endeavors. However, there are processes which are not repeated exactly in the same way twice. This is the case with project-oriented business processes. Such processes, are conducted as projects, in that they are usually planned a priori and run under limited resources. 
	
	\item \textbf{Visualization.} This thesis provides more suitable visualization techniques that help understanding the work process that lies behind software development. More specifically, we presented different possible visualizations to better represent each of the four perspectives of a business process. By mining a GANTT chart, artifact A1 gives insights into the time perspective. By uncovering the hidden co-evolution of software components such a files, artifact A2 shows explicitly the variation of different cases. By mining roles, artifact A3 visualizes de-facto profiles for each developer. By mining activities and their order, artifact A4 gives insights into the control-flow. 
	
	\item \textbf{Extension of the scope of process mining.} Process mining techniques rely on structured data. Typically these data come from \gls{pais}. The artifacts constructed in this thesis handle data which were not generated by \gls{pais}. Thus, they provide a use case for extending the scope of process mining. 
	
	\item \textbf{Datasets for further research.} During the course of this research a number of datasets have been generated. The artifacts produced by this dissertation provide preprocessed datasets and algorithms that can be used for a variety of analyses on the software process. These datasets are available for further use by researchers who want to replicate these studies or pursue different goals using similar concepts to the ones presented in this thesis. 
	
\end{itemize}


\section{Discussion} 
\label{sec:8-discussion}


On a broader context, we discuss the implications of this research for industry and academia. 

For industry, these results of this thesis provide the basis for developing a monitoring tool that allows the manager to obtain insights on software development from a process perspective. The four artifacts can be combined together into a dashboard which gives rich information on the various perspectives of the process. Furthermore, existing and new performance indicators can be included to the ones already presented in this thesis, according to the specific needs of the domain. Such dashboard would provide new insights to help managers in making informed decisions based on facts. 

For academia, this work informs both areas of business process management and software engineering, in particular their sub-fields of process mining and mining software repositories, respectively. Process mining researchers can use data from a new domain, such as software. Researchers from the mining software repositories area can exploit a new lens on analyzing their data, i.e., a process point of view. 


\section{Future Research and Concluding Remarks}
\label{sec:8-future}

There are several ways in which current research can be improved. 

An integrated visualization that brings all four perspectives together would be beneficial to the managers. This would give a clear picture of the impact of each perspective and let the manager decide which of these to prioritize when it comes to improvement. Therefore, the manager has more control over the project and capacity to react to potential issues. 

Qualitative analyses of the impact of the proposed artifacts in an industrial context. User studies should be conducted for each of the artifacts in order to investigate their perceived usefulness and ease of use. With this feedback the artifacts can be adapted and improved. 

Integrate data from different information systems. Data present in \gls{vcs} only store part of the overall end-to-end software development methodology. Integrating data from systems, such as project planning systems, emails, documents, etc, would enable for learning more about how ideas turn into specific process steps. This, in turn, would open up chances for process redesign or innovation. 




