\chapter{Conclusion}
\label{ch9-conclusion}

This chapter summarizes the main contributions of the thesis. \Cref{sec:summary-of-results} lists the results. \Cref{sec:8-discussion} discusses in more detail their impact. \Cref{sec:8-future} outlines future research. 

\section{Summary of Results}
\label{sec:summary-of-results}

This thesis aims at bridging the gap between automatic analysis of software development data and process mining. It provides analyses techniques that help learning specific characteristics of the software process from its trace data. The main results of the thesis are the following.


\begin{itemize}
	\item \textbf{Artifact A1} consisted of an approach to mine the Gantt chart of the software development from commits in \glsfirst{vcs}. With this Gantt chart, the manager can visualize what chunks of work were performed in which part of the process. For example, it is possible for the manager to check whether continuous work was performed at specific work packages in a certain period of time. 
	The approach was implemented as a Java prototype. Its evaluation against real-world datasets from GitHub showed that it is possible to apply this technique to various software projects. Results showed that the  approach is applicable to projects of different size and allows to have both aggregated and detailed views on the development. This kind of information was no possible to retrieve with other mining techniques. Artifact A1 was presented in \Cref{chap:artifactA1} and published in \cite{DBLP:conf/bpm/BalaCMRP15}.
	
	\item \textbf{Artifact A2} consisted of an approach to mine software development data in a higher level of detail by exploiting both quantitative and textual data of \gls{vcs}. With this approach it is possible for the manager to find out pieces of software which have a high coupling. For example, files located at different places in the file systems may be worked on in a similar fashion and at similar times. Artifact A2 refers to this as hidden co-evolution. This can provide a hint to bad modularization practices. For an easier understanding of co-evolution from managers, the developed approach also provides a process model representation of the changes mined from the comments in the \gls{vcs}. This artifact was implemented as an ensemble of several prototypes. Its application on real world data from GitHub showed its applicability and efficacy on identifying hidden dependencies. Artifact A2 was presented in \Cref{chap:articleA2} and published in \cite{DBLP:conf/bpm/BalaRGBMS17}.
	
	\item \textbf{Artifact A3} consisted of an approach to mine user roles from \gls{vcs} comments. With this approach it is possible for the manager to check what are \emph{de facto} roles that people assume when they are working.  These roles may not always be the same in which participants were appointed. In such cases it is up to the manager to re-organize the team in order to best match participants with the most suitable tasks. The approach was developed as an ensemble of several technologies that handle \gls{vcs} data, and offer \glsfirst{nlp} and machine learning algorithms for classification. Its application on a partner company's dataset showed the efficacy of the technique in classifying between different roles such as \emph{developer} or \emph{tester}. This artifact was presented in \Cref{chap:resource-classfication} and published in \cite{DBLP:conf/edoc/AgrawalATBRT16}.
	
	\item \textbf{Artifact A4} consisted of a technique to visualise the various phases of the \glsfirst{sdlc}. With this artifact, the manager can have an overview of where the progress of the software is currently standing and how much effort was put in the different activities of the development life-cycle. The approach was implemented as a Java prototype and can be applied to \gls{vcs} project data. Its application on real world projects from GitHub allowed to discover various patterns of development and calculate several metrics such as the Gini index (i.e., the effort distribution among software developers). Artifact A4 was presented in \Cref{chap:project-mining2} and published in \cite{DBLP:conf/ifip8-1/BalaKM20}.
	
	\item Furthermore, this thesis presented a \textbf{Software Process Evaluation} study. This study analyses the various elements of the \glsfirst{sdm} using a mixed-method. Through questionnaires to the various participants, it identifies the constituting elements of the methodology along with the perceived relevance and performance. Then, through data-analyses techniques (i.e., process mining) it allows for detailed analyses from an objective point of view. The results of this study allowed to develop recommendations to the manager for improving the existing \gls{sdm}. An evaluation with a real-world Austrian company was conducted in which the manager confirmed the relevance of the findings. This work was presented in \Cref{chap:article5} and published in \cite{Vavpotic2022}.
	
\end{itemize}

%\begin{itemize}
%	
%	\item \textbf{Conceptualization of project-oriented business processes.} Processes have been seen so far as repeatable endeavours. However, there are processes which are not repeated exactly in the same way twice. This is the case with project-oriented business processes. Such processes, are conducted as projects, in that they are usually planned a priori and run under limited resources. 
%	
%	\item \textbf{Visualization.} This thesis provides more suitable visualization techniques that help understanding the work process that lies behind software development. More specifically, we presented different possible visualizations to better represent each of the four perspectives of a business process. By mining a GANTT chart, artifact A1 gives insights into the time perspective. By uncovering the hidden co-evolution of software components such a files, artifact A2 shows explicitly the variation of different cases. By mining roles, artifact A3 visualizes de-facto profiles for each developer. By mining activities and their order, artifact A4 gives insights into the control-flow. 
%	
%	\item \textbf{Extension of the scope of process mining.} Process mining techniques rely on structured data. Typically these data come from \gls{pais}. The artifacts constructed in this thesis handle data which were not generated by \gls{pais}. Thus, they provide a use case for extending the scope of process mining. 
%	
%	\item \textbf{Datasets for further research.} During the course of this research a number of datasets have been generated. The artifacts produced by this dissertation provide preprocessed datasets and algorithms that can be used for a variety of analyses on the software process. These datasets are available for further use by researchers who want to replicate these studies or pursue different goals using similar concepts to the ones presented in this thesis. 
%	
%\end{itemize}


\section{Research Questions Revisited} 
\label{sec:8-discussion}

This section revisits the research questions introduced in \Cref{sec:problem-definition} and discusses how this thesis addressed each of them. More specifically the general question this thesis aims at answering is the following.
\begin{question}{Main Research Question}
	\textbf{RQ}: \emph{How can we make use of project event data to gather insights about the software development process that are informative to managers?}\\
	 
	Trace data can be exploited to obtain a more realistic evaluation of the current state of the software project. Among the articles presented in this thesis, Article 5 presents a case-study where the importance of trace data is highlighted. While there have been many approaches that try to assess the software process (e.g., \citep{DBLP:journals/bise/VavpoticRH20,hovelja2015exploring,atkinson1999project}) with various methods, there was no approach that exploits both \emph{subjective} data gathered from questionnaire and \emph{objective} data from software development. Thus, the case study presented in Article 5 implements a one-of-a-kind data-driven computationally-intensive approach \citep{DBLP:journals/isr/BerenteSS19} to gather insights the software process. 
	In particular, key difficulties in the performance of the \gls{sdlc} of the analysed company were discovered. These results were only possible thanks to the combination of the subjective views of process participants and the analyses of the event logs. The latter served to confirm or dis-confirm the importance of certain \gls{sdm} elements with regards to performance.
	
\end{question}

The main research is addressed in detail by applying mining techniques to event data from software projects. In particular, this thesis applies the four-perspectives view on processes defined in the work of \cite{DBLP:books/sp/Aalst16}. Consequently the main research question is divided into four sub-questions, each addressing one specific perspective.  In the following, each research sub-question \textbf{RQ1} – \textbf{RQ4} is discussed.\\


\begin{question}{Mining the Time Perspective}
\textbf{RQ1.} \emph{How can we use project event data to extract information about the \textbf{temporal} perspective of activities?}\\


RQ1 is addressed by Artifact A1. The context in which this artifact was developed is that of large engineering software development projects. These projects need to be monitored in detail to understand what work was done \emph{when}. This is particularly relevant when it comes to checking compliance with rules and regulations such as ISO/IEC/IEEE 15288 \citep{international2015iso}. A typical problem of these processes is coordination as it is difficult to tract the actual course of work, even if the data is recorded in \glsfirst{vcs}. Artifact A1 addresses this problem by defining a mining technique that is able to handle data from \gls{vcs}. Furthermore, Artifact A1 provides a time-based visualisation that allows to cluster together periods of work that were done in the different parts of the software directory structure. The models are rendered as Gantt charts in order to best address project managers, as these are directly comparable to projects plans. The artifact was implemented as a prototype, which gives the user both aggregated information in the form of Gantt chart activities and detailed information on the single events. This improves the information given by similar process mining software such as the Dotted Chart Miner \citep{Song2007}. Moreover, no existing process mining tool gives Gantt chart support.
\end{question}



\begin{question}{Mining the Case Perspective}
	\textbf{RQ2.} \emph{How can we use project event data to extract information about the \textbf{case} perspective?}\\
	
	
	Mining the case perspective in the software development process signifies to uncover characteristics of cases (i.e., the different ways in which files are worked on). This is difficult because work in such process is fragmented between different modules which are developed concurrently. Prior contributions in the areas of \glsfirst{msr} and process mining have proposed a plethora of techniques that analyse and visualize the current state of a software as a product. Surprisingly, a process view (i.e., what type of work is conducted at different stages and what are activity dependencies) was still lacking. Artifact A2 addresses such gap via a technique that can extract the evolution of each file as a time series (i.e., it provides a characterisation of the changes of the file as a process case) and it further classifies the type of work (i.e., the process activities) based on \gls{nlp} techniques. Efficacy of the devised Artifact A2 is demonstrated by applying the technique on various open-source projects. 
\end{question}





\begin{question}{Mining the Organisational Perspective}
	\textbf{RQ3.} \emph{How can we use project event data to extract information about the \textbf{organizational} perspective?}\\
	
	
	Collaboration in business processes and projects requires a division of responsibilities among the participants who are then expected to perform specific tasks based on their role in the organisation. However, in practice process participants are free to perform software developments tasks. The type and amount of tasks that are done by the participants define a so-called \emph{profile}. Profiles are \emph{de-facto} roles may differ from the assigned roles by the project manager. Artifact A3 defines a technique to collect participants' profiles from \gls{vcs} logs and classify them into classes of roles. It does so via two approaches, which are then compared. The first approach finds classes of users by applying k-means clustering to users based on attributes calculated for them. The classes identified by the clustering are then used to build a decision tree classification model. The second approach classifies individual commits based on commit messages and file types. The distribution of commit types is used for creating a decision tree classification model. The two approaches are implemented and tested against three real datasets, one from academia and two from industry. Our classification covers 86\% percent of the total commits. The results are evaluated with actual role information that was manually collected from the teams responsible for the analysed repositories. In practice, the developed artifact helps the project manager to better understand the actual type of work that is required in developing a specific software product.
\end{question}

 


\begin{question}{Mining the Control-Flow Perspective}
	\textbf{RQ4.} \emph{How can we use project event data to extract information about the \textbf{control-flow} perspective?}\\
	 
	
	RQ4 is addressed by Artifact A4. This artifact is developed with the assumption that the complexity of software development process is hard to be discovered as a single process model. In fact, existing models of the \gls{sdlc} always remain at a high level of abstraction and never specify exactly the low-level activities that constitute the different phases of software development. Moreover, many \glspl{sdlc} models assume that certain activities (or phases) of software development happen in a specific order and never overlap. An exception to this is the so-called \glsfirst{rup} model, which acknowledges that different phases of the development can be done concurrently. However, the \gls{rup} has only been a theoretical model so far and no tools were developed to mine such model from data. Artifact A4 was developed to fulfil this goal. It uses data from \gls{vcs} to analyse to mine the activity types of which the development process consists. The artifact is implemented as a prototype in Java and its outputs evaluated in terms of effectiveness against existing real-world repositories from GitHub. As a result, various patterns of software development were found and visualized according the \gls{rup} model. In this way, it was possible to understand the how the work transitioned among the different phases (i.e., the control-flow) of software development. 
\end{question}

\section{Future Research and Concluding Remarks}
\label{sec:8-future}

There are several ways in which current research can be improved. 

\begin{itemize}
	\item {\bfseries Develop more advanced visualisation techniques.}
	Visualisation is key to process understanding \citep{DBLP:journals/corr/abs-2202-07941}. An interesting follow-up of this research would be to provide an integrated visualization that brings all four perspectives together in a way that would be beneficial to the managers. This would give a clear bigger picture of the impact of each perspective and let the manager decide which of these to prioritize when it comes to improvement. Therefore, the manager has more control over the project and capacity to react to potential issues. 
	
	\item{\bfseries User studies.}
	Qualitative analyses of the impact of the proposed artifacts in an industrial context. User studies should be conducted for each of the artifacts in order to investigate their perceived usefulness and ease of use \citep{DBLP:journals/misq/VenkateshMDD03}. With this feedback the already developed artifacts can be improved. 
	
	\item{\bfseries Integrate different datasets.}
	Another point concerns the data used for the analysis. This thesis used data from one source at a time such \gls{vcs} or Jira. However, data present in these systems only store part of the overall end-to-end software development methodology. Integrating data from diverse systems \citep{DBLP:conf/icse/TrautschTHLG20}, such as project planning systems, emails, documents, etc, would enable for learning more about the overarching software process (e.g., the end-to-end process from idea-inception to software-product release). This, in turn, would open up chances for process redesign or innovation. 
	
	\item{\bfseries Coordination studies.}
	Coordination is an important topic when it comes to software development. Future work shall investigate how control happens in \gls{oss}. Especially, it should aim at shedding light on the how collaborative processes unfold when several users are free to join existing software projects as contributors. \Gls{floss} projects have been brought as an example of innovative ways of self-organizing \citep{DBLP:journals/mansci/KroghH06,DBLP:journals/misq/HowisonC14}. Works in literature have emphasized iterations (\citealp{Berente2005,Berente2007}), self-organization \citep{DBLP:journals/infsof/CrowstonLWEH07,DBLP:journals/jss/HodaM16} and free-speech \citep{DBLP:conf/chiir/ThomasCMCM18,Gibson2019}. However, it has been shown that control mechanisms (be there social or institutional) are present in the area \citep{Lindberg2016}. In particular, online collaborative work that generates knowledge goes through ``negotiation'' actions which ``pull'' the trajectory and shape its movement in the feature space \citep{Arazy2020}. Therefore, it remains under-explored to what extent \gls{oss} is institutionalized and what are its similarities to corporate bureaucracy.
	
	A possible way to approach the problem of freedom in open source development is to compare it with institutionalised practices, such as corporate bureaucracy. Future work shall analyse pull requests from a real-world open source repository using speech act theory \citep{searle1985expression}. This, in combination with data analysis techniques and process mining, would allows us to untangle interesting insights about OSS development and point out relevant resemblance to corporate processes (e.g., stage-gate  \citep{cooper2008perspective}). 
	
	
	\item{\bfseries New forms of organising: Holacracy.}
	Another future direction is to apply principles developed in this thesis to data from new forms of organisations  such as Holacracy. An initial dataset of a real-world Holacratic organisation has already been constructed (see \citep{Wurm2021}) which contains five years of trace data. These data can be used to mine various patterns of collaboration and division of influence partially exploiting approaches of the already presented artifacts A1 – A4 and including further analyses such as dynamic networks \citep{DBLP:journals/csur/RossettiC18,DBLP:journals/corr/abs-0803-2093}. Possible outcomes would be community discovery and their comparison to the actual communities (circles) that are defined in the organisation.
	
	\item{\bfseries Process mining on source code.}
	As logs are highly relevant to maintenance and post-mortem analyses of software, it is of utmost importance that developers place the correct logging commands within methods. This issue is referred to as the  \emph{where-to-log} problem in the \gls{msr} field.  While many approaches have tackled such problem \citep{DBLP:conf/icse/FuZHLDLZX14,DBLP:conf/icse/ChenJ17,DBLP:conf/msr/CandidoHAD21}, not many approaches exist that apply process mining to source code. More specifically, approach shall get as input the source code from GitHub and parse its \gls{ast}. Different commands used in the programming language shall be mapped to process activities. By analysing different variant of these activities, possible patterns of logging choices shall emerge. 
	
	
\end{itemize}

This thesis has implications for academia and industry as follows. 
For industry, the results of this thesis provide the basis for developing a monitoring tool that allows the manager to obtain insights on software development from a process perspective. The four artifacts can be combined together into a dashboard which gives rich information on the various perspectives of the process. Furthermore, existing and new performance indicators can be included to the ones already presented in this thesis, according to the specific needs of the domain. Such dashboard would provide new insights to help managers in making informed decisions based on facts. 

For academia, this work informs both areas of business process management and software engineering, in particular their sub-fields of process mining and mining software repositories, respectively. Process mining researchers can use data from a new domain, such as software. Researchers from the mining software repositories area can exploit a new lens on analysing their data, i.e., a process point of view. 

