\chapter{Research Background}
\label{ch2:problem-background}

\section{Software Process}
\label{sec:ch2-software-process}

What is the software process. Humphrey: (rationally) Managing the software process?

\section{Theorizing the Software Process}
\label{sec:ch2-theorizing}

\subsection{Organization Routines}
\label{subsec:org-stud}

\subsection{Coordination Theory}

\subsection{Business Process Management}

\subsection{Language-Action (Perspective)}

Winograd \& Flores~\citep{Winograd1986a}

\section{Tool Support for Software Development Processes}
\label{sec:ch2-tool-support}

Talk about what tools are used.
Especially what tools/repositories I analyze. 



\section{Software Process Monitoring }
\label{sec:ch2-sw-p-monitoring}

\subsection{What is important to be monitored}

\subsection{How far it is difficult}

\section{Research Questions}
\label{sec:ch2-research-questions}

Summarize the gap here. Make it clear that the bigger research question is divided into 4 research questions. It is split on perspective. 
Refer to the main RQ as research problem (RP). Then you have RQ1, RQ2, ... .


Before formulating the research questions, build the context with the terminology. Kate says "RQ1. Which activities and when?".

\todo[inline]{Terminology must be defined somewhere. Maybe we need a preliminaries section. For example, what is an artifact? Define it.}

Addressed in the main chapters.

Type of work: testing, development, documentation, bug-fixing, feature specification. 

Assumption is that the project is structured in folders. A project is broken down into parts. Parts can be mapped to folders.

RQ: How can we make use of project event data to gather insights about the software development process that are
informative to managers

\begin{description}
	\item[RQ1] What type of work was performed? 
	(Paul's thesis addresses this.)
	\item[RQ2] How were activities performed in the different parts of the project?
	\item[RQ3] description
\end{description}

