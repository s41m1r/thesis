\chapter{Mining Project-Oriented Business Processes}

\todo[inline]{Here goes BPM1}

The class of processes that we discuss in this chapter are long-term engineering
projects. These processes have specific requirements for monitoring. First, they
are executed only once according to the specific needs of a particular project, and
only partially according to recurring process descriptions. Second, they involve
various actors that typically document their work in a semi-structured way using
text and tables. Third, work in the project is usually subject to constraints
regarding the start and end and the temporal order. Fourth, there is typically
no process engine controlling the execution. Fifth, even though these limitations
in terms of traceability exist, there are usually strong requirements in terms of
tracking when which work was conducted.


Here goes the contribution from BPM1. 

\section{Project-Oriented Business Processes}
\todo[inline]{Rephrase here}

A project-oriented business process can be
defined as an ad-hoc plan that specifies the tasks to be performed within a limited
period of time and with a limited set of resources for achieving a specific goal.
Unlike repetitive business processes for which notations such as BPMN [12] or
EPC [1] are commonly used, project-oriented business processes may be properly
represented with PERT or GANTT models.

\section{Approaches to Mine Project-Oriented Business Processes}
\todo[inline]{Rephrase here}

The problem described has been addressed in the literature from different per-
spectives. The first category of related work tackles the problem by transforming
it into a process mining problem. Consequently, approaches have been developed
to preprocess VCS data such that process mining techniques can be applied, and
hence, a business process can be derived from the log data. In this group, Kindler
et al. [9,10] developed an algorithm for extracting software processes that are
mapped to Petri Nets. Activities, which are not explicit in the logs, are discovered
from their input and output artifacts. However, strong assumptions are made on
the filenames as well as on the software process lifecycle. Rubin et al. in [15] ad-
dressed the problem of engineering processes that are not well documented and
are usually unstructured. They provided a bridge from Kindler et al.’s approach
to ProM [5] in order to mine different process perspectives, such as performance
social network analyses. Rubin et al. [16] applied process mining to the touristicMining Project-Oriented Business Processes
5
industry and obtained user processes from web client logs pursuing the goal of
improving the software system by analyzing the underlying process. Poncin et
al. [14] developed the FRASR framework for preprocessing software repositories
to transform the VCS data to logs that conform to the process mining event log
meta model [4] as utilized in ProM [5]. However, these approaches disregard the
single-instance nature of project-oriented business processes and treat them as
procedures that can be repeated over time.
The second category of related work focuses on the visualization of VCS data
for different purposes. Several approaches study the interaction among develop-
ers over time from a visualization point of view. For instance, Ogawa and Ma
[11] drew storyline pathways to show the story of each developer’s contribution.
Other approaches analyze and visualize VCS data at file level in order to discover
file version evolution. Voinea and Telea [20] introduced an interactive navigation
method to surf file version evolution as well as two methods to cluster versions
of the same file in an abstraction layer. Wu et al. [22] also visualized the evolu-
tions of entire projects at file level, emphasizing the evolution moments. Finally,
several approaches study change prediction with the aim of discovering predic-
tion patterns that can help in the process of software development [24,23]. The
approaches mentioned in this category as well as others that apply similar tech-
niques [6,8,3] focus on studying software evolution from different standpoints.
However, the goal pursued differs in all cases from our goal in that they are not
interested in discovering projects tasks out of the log data, and hence, they lack
an explicit notion of work structure that we need to consider for our purpose.
Our approach combines ideas from both areas, as we aim at identifying tasks
like in the approaches that rely on process mining, but we must cluster the data
in an appropriate way, for which techniques developed in the approaches that
pursue visualization may be adapted or extended.